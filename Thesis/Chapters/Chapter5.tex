% Chapter Template

\chapter{Conclusions} % Main chapter title

\label{Chapter5} % For referencing this chapter elsewhere, use \ref{Chapter5}

%----------------------------------------------------------------------------------------
%	SECTION 1
%----------------------------------------------------------------------------------------

\section{Results analysis and solutions quality}

In the last section the most significant results have been shown. However, some specificities of each algorithm that we may not observe in those figures must have been explained for a better comprehension of the behaviour, advantages and disadvantages of each algorithm.

Regarding the Local-Global policy, it must be said that there are many solutions quality aspects included in the schedule optimization it performs apart from how many tasks have been finally scheduled (remember section \ref{sec_F_LG}). Moreover, the tasks can be scheduled concurrently and any resource definition can be introduced in the problem definition. Finally, the golden index can be optimized for each satellite, either statically or dynamically by the \emph{Global} instance. This can model much better the heterogeneity among the satellites. In fact, homogeneity in the system deteriorates the results of the Local-Global, as very similar sub-solutions are delivered to the \emph{Global} from several satellites, giving no chance to find a good combination.

The price-based scheduler has also some characteristics that are not found on Local-Global: dependence between tasks and accounting the communication cost of passing the results from one task to another is included in the schedule calculation. Furthermore, the market model allows us to redefine the bid calculation for taking into account whichever parameter we want to adapt the performance to a particular system. This could be done also either statically or dynamically.

When both algorithms are compared, we find that the price-based execution times and memory usage is lower than the Local-Global's, but there are some handicaps to be observed: the price-based ``sequentilaization'' of the scheduling process is a bottleneck and causes a saturation point that decreases the performance for high number of input tasks, while the Local-Global's saturation is softer and it achieves to maintain the number of scheduled tasks for this situation. Price-based scheduler also requires good communication among the satellites forming the cluster, something not very realistic in a highly-constrained bandwidth context such as the open space. The Local-Global's poor performance for high values of number of satellites and golden index can be improved by optimizing the \emph{Local} entity and smartly adapting the values of the golden index.

\section{Future work}

In this work the Local-Global policy's design has been completed and fully implemented, finding out and implementing the most appropriate state-of-the-art system for comparing both and test the real Local-Global's performance. The parameters fitting most to the comparison's goals have been defined and large testing benchmarks have been performed for both algorithms, quantitatively measuring the behaviour of each one of them.

These complete work allows us to firmly state some conclusions about the Local-Global policy's future work to be done:
\begin{itemize}
\item As it has already been said, although the current performance can be considered as a very good result for the first approach, an improvement can be obtained by optimizing the \emph{Local} scheduler. The optimization of this scheduler should be done on the sub-solutions generation: although each sub-solution, if considered on its own, is a good result, obtained in very low time, the whole set of $\Delta_i$ sub-solutions are not optimal, as they may not be the best sub-solutions to be found, and the may contain repetitions among them.
\item Also some optimizations could be included in the \emph{Global} combinatorial search. Despite having very good performance when heterogeneous satellites form the cluster, for homogeneous cases it decreases dramatically. These are the cases to be studied for the improvements to be carried in the algorithm.
\item Task dependence defined in the price-based scheduler could be included in the policy, as it is a practical functionality to be implemented for small tasks forming an entire mission or activity.
\end{itemize}

To conclude, the results obtained in this Bachelor Thesis have allowed to characterize a theoretical sketched distributed task scheduler that could be implemented as a critical part of the autonomy system of future fractionated satellite systems.